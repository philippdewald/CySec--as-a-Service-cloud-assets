\documentclass[12pt]{article}
\usepackage[utf8]{inputenc}
\usepackage[margin=1in,bottom=1.5in,a4paper]{geometry}
\usepackage{tikz}
\usepackage{amsmath}
\usepackage{amssymb}
\usepackage{multicol}
\usepackage{xcolor}
\usepackage{tabularx}
\usepackage{mathtools}
\usepackage{ textcomp }
\usepackage{graphicx}
\usepackage{ stmaryrd }
\usepackage{hyperref}
\usepackage{ marvosym }
\usepackage{ dsfont }
\usepackage{ulem}
\tolerance=1000
\usepackage{fancyhdr}
\pagestyle{fancy}
\headheight 50pt

\renewcommand{\thesection}{\Alph{section}}

%edit header and footer
\fancypagestyle{firstpage}{
  \lhead{X-as-a-Service cloud assets}
  \chead{\textbf{\Large cloud-detection.py}}
  \rhead{CySec Project '21}
}
\lhead{}
\rhead{X-as-a-Service cloud assets}

\cfoot{}
\rfoot{\small\thepage}

\begin{document}
\thispagestyle{firstpage}

\begin{center}
    This document describes step by step what the \verb|cloud-detection.py| does. \\
\end{center}
The first draft will only conclude detection of Azure, however, implementing other cloud providers should be added easily.
\begin{itemize}
    \item Resolve the IP 
    \item Check the nameservers
    \item Check the MX, TXT, ... records
    \item Check the certificates
    \item Check the \verb|server| attribute in the Response Header
    \item Check the Autonomous System
    \item 
\end{itemize}

\end{document}